%% Preamble
%% ----------------------------------------------------------------------
\documentclass[12pt, twoside, a4paper, final]{article}
\usepackage[some]{background}
\usepackage[utf8]{inputenc}
\usepackage[spanish]{babel}
\usepackage[T1]{fontenc}
\usepackage{pgfcalendar}
\usepackage{pdflscape}
\usepackage{colortbl}
\usepackage{enumitem}
\usepackage{fancyhdr}
\usepackage{graphicx}
\usepackage{geometry}
\usepackage{hyperref}
\usepackage{multirow}
\usepackage{pgfplots}
\usepackage{tabularx}
\usepackage{titlesec}
\usepackage{charter}
\usepackage{hologo}
\usepackage{lipsum}
\usepackage{xcolor}
\usepackage{array}
\usepackage{avant}

%% Settings
%% ----------------------------------------------------------------------
%% Hyperref package settings
\hypersetup{
    pdftitle={Prueba de estrés y rendimiento para el Servidor de Microsoft Dynamics NAV\textregistered. Instalaci\'{o}n de Norma},
    pdfauthor={Mart\'{i}n Josemar\'{i}a Vuelta Rojas},
    pdfpagelayout=OneColumn,
    pdfnewwindow=true,
    pdfdisplaydoctitle=true,
    pdfstartview=XYZ,
    plainpages=false,
    unicode=true,
    bookmarksnumbered=true,
    bookmarksopen=true,
    bookmarksopenlevel=3,
    breaklinks=true,
    colorlinks=true,
    pdfborder={0 0 0}
}

%% Red square in the title page og the document
\backgroundsetup{
    scale=1,
    angle=0,
    opacity=1,
    contents={
        \begin{tikzpicture}[remember picture,overlay]
            \path [fill=red-softbutterfly] (current page.north west) rectangle (-8.5, 0);
            \path [fill=blue-softbutterfly] (current page.south west) rectangle (-8.5, 0);
        \end{tikzpicture}
    }
}

%% Customizing title formats for chapter (use of sans serif font family)
\titleformat{\title}[display]
    {\normalfont\rmfamily\Huge\bfseries\color{grey-800}}
    {}{0pt}{}

%% Customizing title formas for sections (use of sans serif font family)
\titleformat{\section}
    {\normalfont\sffamily\large\bfseries\color{grey-800}}
    {\thesection}{1em}{}

%% Customizing header spacings in page geometry
\geometry{
    a4paper,
    headheight=25pt,
    headsep=12pt,
}

%% Pgfplots settings
\pgfplotsset{compat=1.14}
\usepgfplotslibrary{dateplot}
\usepgfplotslibrary{groupplots}


%% Command definitions
%% ----------------------------------------------------------------------
\makeatletter

%% Color definitions
\definecolor{red-softbutterfly}{rgb}{0.7843137254901961, 0.21568627450980393, 0.21568627450980393}
\definecolor{blue-softbutterfly}{rgb}{0.08627450980392157, 0.17647058823529413, 0.3137254901960784}
\definecolor{red-A700}{rgb}{1.0000, 0.0000, 0.0000}
\definecolor{grey-050}{rgb}{0.9804, 0.9804, 0.9804}
\definecolor{grey-800}{rgb}{0.2588, 0.2588, 0.2588}
\definecolor{grey-900}{rgb}{0.1216, 0.1216, 0.1216}

\newcommand{\globalcolor}[1]{\color{#1}\global\let\default@color\current@color}


%% parragraph spacings
% \setlength{\parindent}{4em}
% \setlength{\parskip}{0.5em}

%% Table columns types
\newcolumntype{L}[1]{>{\raggedright\let\newline\\\arraybackslash\hspace{0pt}}m{#1}}
\newcolumntype{C}[1]{>{\centering\let\newline\\\arraybackslash\hspace{0pt}}m{#1}}
\newcolumntype{R}[1]{>{\raggedleft\let\newline\\\arraybackslash\hspace{0pt}}m{#1}}

%% Serif type font as default for all document
\renewcommand{\familydefault}{\rmdefault}

%% Clear double page with the standar text "This page is intentionally left in blank" translated to spanish
\def\cleardoubleintentionalpage{
    \clearpage\if@twoside%
    \ifodd\c@page\else
    \vspace*{\fill}
    \begin{center}
    % Use this for spanish
    \rm\emph{Página intencionalmente dejada en blanco.}
    % Use this for english
    % \rm\emph{This page is intentionally left in blank}
    \end{center}
    \vspace{\fill}
    \thispagestyle{empty}
    \newpage
    \if@twocolumn\hbox{}\newpage\fi\fi\fi
}

%% Completely clean pages
\def\cleardoublepage{
    \clearpage\if@twoside%
    \ifodd\c@page\else
    \vspace{\fill}
    \thispagestyle{empty}
    \newpage
    \if@twocolumn\hbox{}\newpage\fi\fi\fi
}

%% Flip margins
\newcommand*{\flipmargins}{%
    \clearpage
    \setlength{\@tempdima}{\oddsidemargin}%
    \setlength{\oddsidemargin}{\evensidemargin}%
    \setlength{\evensidemargin}{\@tempdima}%
    \if@reversemargin
        \normalmarginpar
    \else
        \reversemarginpar
    \fi
}

%% Header and footer styles
%% \def\unionlabellogo{\includegraphics[width=2.5cm]{./uln-logo.pdf}}
%% \def\microsoftnavlogo{\includegraphics[width=3.7cm]{./nav-logo.png}}
\fancypagestyle{unionlabel}{
    \fancyhf{}
    \renewcommand{\headrulewidth}{1pt}%
    \renewcommand{\footrulewidth}{0pt}%
    %% \fancyhead[EL]{
    %%   \parbox[b]{2.5cm}{\unionlabellogo}%
    %% }
    %% \fancyhead[OR]{
    %%   \parbox[b]{3.7cm}{\microsoftnavlogo}%
    %% }
    \fancyfoot[C]{
      {
        \begin{centering}
          \thepage
        \end{centering}
      }
    }
}

%% Registered mark text
\newcommand{\registeredmark}{\textsuperscript{\small{\textregistered}}}
\makeatother


%% Document
%% ----------------------------------------------------------------------
\AtBeginDocument{\globalcolor{grey-900}}
\begin{document}
    %% Front Page -------------------------------------------------------
    \begin{titlepage}
        \BgThispage
        \pagecolor{grey-050}
        \newgeometry{left=5cm,top=6cm,bottom=6cm,right=3.5cm} %defines the geometry for the titlepage
        {\noindent\includegraphics[width=7cm]{../resources/SoftButterfly-LaTeX-Logo.pdf}\par}
        {\noindent\color{red-softbutterfly}\makebox[0pt][l]{\rule{1.3\textwidth}{1pt}}\par}
        % {\noindent\raggedright
        %     \color{grey-800}\large
        %     \textit{Curso}\par}
        \vfill
        {\noindent\raggedright
            \color{grey-800}\huge
            \textsf{Preparación de documentos con} \LaTeX\par}
        \vfill
        {\noindent
            \color{blue-softbutterfly}
            \textsf{Sílabus elaborado por}\par}
        {\noindent
            \color{grey-800}
            \large\textsf{Martín Josemaría Vuelta Rojas}\par}
        \vskip\baselineskip
        {\noindent
            \color{grey-800}
            \textsf{Lima, \today}\par}
        \afterpage{\cleardoubleintentionalpage}
    \end{titlepage}
    \restoregeometry
    \pagecolor{grey-050}

    %% Tables of contents -----------------------------------------------
    % \flipmargins
    % \tableofcontents
    % \cleardoubleintentionalpage

    % \listoffigures
    % \cleardoubleintentionalpage

    % \listoftables
    % \cleardoubleintentionalpage

    %% Page style -------------------------------------------------------
    %% Report content ---------------------------------------------------
    {\noindent\raggedright
        \color{grey-800}\huge
        \bfseries\textsf{Preparación de documentos con} \LaTeX\par}
    \vspace{3em}
    \section{Introducción}
        Curso de nivel intermedio, con una introducción básica, elaborado para afianzar los conocimientos y técnicas para una redacción efectiva en {\LaTeX} y las directrices para una correcta aplciacion de estilos y formatos para diversos tipos de publicaciones, desde artíclos académicos hasta libros y revistas.

        Dada la fuerte presencia de {\LaTeX} en el entorno académico este curso hace énfasis y provee las herramientas necesarias para lograr artículos y revistas académicas de calidad, y para la elaboración de notas de clases y exámenes, siendo ideal para estudiantes y profesores que deseen iniciar en {\LaTeX} o afianzar sus conocimientos previos.

    \section{Detalles}
        \begin{description}[style=nextline]
            \item[Nivel] Intermedio.
            \item[Método] Teórico - Práctico.
            \item[Duración] 16 horas\footnote{Se sugiere una división en 4 sesiones de 4 horas con un receso de 15 mintuos en cada sesión.}
            \item[Publico objetivo] Estudiantes y profesores que deseen iniciar en {\LaTeX} o afianzar sus conocimientos previos.
        \end{description}

    \section{Temario}
    \begin{description}
        \item[Sesión 1: Introducción] \hfill
            \begin{enumerate}
                \item Introduccion: {\LaTeX} y {\TeX}.
                \item Instalación: Recomendaciones para Linux, Windows y Mac.
                \item Estrcutura de un documento en {\LaTeX}.
                \item Compilación.
                \item Sintaxis básica: Grupos, entornos y comandos.
                \item Formato de página.
                \item Formato de texto, títulos y párrafos.
                \item Colores, Fuentes y Caracteres especiales.
                \item Soporte de idiomas.
                \item Listados.
                \item Tablas.
                \item Entornos flotantes, figuras and leyendas.
            \end{enumerate}

        \item[Sesión 2: Referencias, matemáticas y programación] \hfill
            \begin{enumerate}
                \item Notas a pie de página y al margen.
                \item Enlaces, etiquetas y referencias cruzadas.
                \item Bibliografía incrustada.
                \item Manejo de bibliografías con \hologo{BibTeX}.
                \item Estilos y formatos para bibliografía.
                \item Índices: contenidos, figuras y tablas.
                \item Matemáticas básicas con {\LaTeX}.
                \item Matemáticas avanzadas \hologo{AmSLaTeX}.
                \item Macros en {\LaTeX}.
            \end{enumerate}

        \item[Sesión 3: Generación de gráficos] \hfill
            \begin{enumerate}
                \item Gráficos procedurales
                \item PGF/TikZ
                \item Sintaxis básica
                \item Gráficos en 2D
                \item Formas
                \item Funciones
                \item Carga de datos externos
                \item Gráficos en 3D
            \end{enumerate}

        \item[Sesión 4: Tipos de documentos y buenas prácticas] \hfill
            \begin{enumerate}
                \item Tipos de documentos
                \item Artículos
                \item Reportes
                \item Libros
                \item Buenas prácticas para el desarrollo de documentos
                \item Modularidad
                \item Archivos de estilo
            \end{enumerate}
    \end{description}
\end{document}
